\documentclass[]{article}
\usepackage{fullpage}
\usepackage{graphicx}
\usepackage{subcaption}
\usepackage{amsmath}
%opening
\title{Magnetohydrodynamic Cocktail Stirrer}
\author{Carlos Gross Jones}

\begin{document}

\maketitle

\begin{abstract}

\end{abstract}
\newpage
\section{Background}
\par The first attempt at a contactless cocktail was based on a simple application of the Lorentz force. Two copper electrodes placed in a glass tumbler passed a current through the liquid, while a strong magnet (a 2''x1''x3/4'' N45 block from United Nuclear Scientific LLC) under the tumbler provided an orthogonal magnetic field (Fig. \ref{fig:oldversion}). 
\begin{figure}
	\centering
	\includegraphics[width=0.5\textwidth]{Oldversion}
	\caption{First iteration of contactless stirrer}
	\label{fig:oldversion}
\end{figure}
Assuming time-invariance, the force applied on a differential element $\mathrm{d}V$ in a current field $\vec{J}$ and an orthogonal magnetic field $\vec{B}$ can be found from the element length d$\ell$ parallel to the current and cross-sectional area d$A$ normal to the current:
\begin{align}
\vec{F}=q\vec{V}\times\vec{B}\\
\vec{F}=I\vec{\ell}\times B\\
I=\lvert J\rvert\mathrm{d}A\\
\end{align}
(Valid because d$A$ is defined to be normal to $\vec{J}$.)
\begin{align}
\mathrm{d}\vec{F}=\lvert\vec{J}\rvert\mathrm{d}A\mathrm{d}\ell\times\vec{B}
\end{align}
Since, in this case, the magnetic and current fields are orthogonal, and since the current field is approximately uniform between the electrodes (simply the total current divided by the electrode area),
\begin{align}
\mathrm{d}\vec{F}=\lvert\vec{J}\rvert\lvert\vec{B}\rvert\mathrm{d}A\,\mathrm{d}\ell\\
\vec{F}=\iiint\limits_V\lvert\vec{J}\rvert\lvert\vec{B}\rvert\mathrm{d}A\,\mathrm{d}\ell
\end{align}
Given an electrode area of $A$ and spacing of $L$ and approximating both the current and magnetic field to be uniform between the electrodes,
\begin{align}
\vec{F}=\lvert\vec{B}\rvert L\iiint\limits_A\lvert\vec{J}\rvert\mathrm{d}A=\lvert\vec{B}\rvert IL
\end{align}
\par This design did work in principle; the Lorentz force produced a pumping action, where fluid would flow down the centerline between the electrodes and recirculate along the walls of the tumbler. However, since a large current was flowing through a fluid, significant electrolysis occurred, which, in addition to producing potentially dangerous hydrogen and oxygen gas, affected the taste of the cocktail due to the electrolysis products of ethanol.

\section{Theoretical Background}
\par Since the basic application of Lorentz force to pumping was effectively validated (and is in fact well studied \cite{yamato}), the major problem remaining was the electrolysis of the cocktail. As per Faraday's Law of Electrolysis:
\begin{equation}
\dot{m}=\frac{I}{F}\frac{M}{z}
\end{equation}
Where:
\begin{itemize}
	\item $\dot{m}$ is the mass of electrolysis products appearing at an electrode per unit time;
	\item $I$ is the total current flowing into or out of the electrode;
	\item $F$ is the Faraday constant, 96485 mol/C;
	\item $M$ is molar mass of the original substance;
	\item $z$ is the number of valence electrons of the substance.
\end{itemize}
\par Since $F$ is a constant and $M$ and $z$ are properties of the substance, in order to minimize $\dot{m}$, $I$ must be minimized. However, pumping force is also directly proportional to $I$. The central proposal of this project is that currents which circulate entirely within a fluid will not result in electrolysis, because the \textit{net} current flow through the fluid is zero. 
\par The obvious question is how to produce currents without an external EMF source such as a battery. The proposed solution is to use an external, time-variant magnetic field to induce circulating currents (eddy currents) in a conductive fluid.

\par For the purposes of a first-pass analysis, the following model will be considered:
\begin{itemize}
	\item The fluid will be a 3'' diameter, 3'' tall cylinder (based on an approximate cocktail tumbler);
	\item The fluid will be considered to have electrical resistivity $\rho$ and magnetic properties equal to a vacuum (water is in fact weakly diamagnetic, but this is expected to be a negligible contribution);
\end{itemize}

\par Since, at this point, the goal is to find the form of the current field and its dependence on $B_{s,max}$ and $\rho$, it is not necessary to accurately know the resistivity of an actual cocktail; those values are calculated later in \S\ref{sec:resistivity}.

\subsection{General Design}
\par Therefore, large permanent magnets were next considered. 3'' dia. x 1'' thick NdFeB45 magnets from United Nuclear Scientific LLC were chosen based on field strength per dollar (\$140 and maximum $\vec{B}\cdot\vec{n}$ of $\approx$0.42 T).

\section{Rotor Design}

\section{Property Measurements}
\subsection{Cocktail Resistivity}
\label{sec:resistivity}
\subsection{Rotor Field Measurement}



\begin{thebibliography}{3}
	\bibitem{yamato}
	Takezawa, Setsuo et al. ``Operation of the Thruster for Superconducting Electromagnetohydrodynamic Propulsion Ship YAMATO 1.'' Bulletin of the M.E.S.J 23.1 (1993): 46-55. Japan Institute of Marine Engineering. Web. 21 Aug. 2016. \textless http://www.jime.jp/e/publication/bulletin/english/pdf/mv23n011995p46.pdf>.\textgreater
	
	\bibitem{curesistivity}
	Matula, R. A. ``Electrical Resistivity of Copper, Gold, Palladium, and Silver.'' \textit{Journal of Physical and Chemical Reference Data} 8.4 (1979): 1161. Web. 23 Aug. 2016. \textless http://www.nist.gov/data/PDFfiles/jpcrd155.pdf\textgreater. 
	
	\bibitem{analyticeddy}
	Bowler, John R., and Theodoros P. Theodoulidis. ``Eddy Currents Induced in a Conducting Rod of Finite Length by a Coaxial Encircling Coil.'' Journal of Physics D: Applied Physics 38.16 (2005): 2861-868. Web. 23 Aug. 2016. 
\end{thebibliography}
\end{document}
